\documentclass{article}
\usepackage{graphicx} % Required for inserting images
\usepackage[smartEllipses]{markdown}
\usepackage{geometry}

% Set the text width
\geometry{
    a4paper,
    left=2cm,
    right=2cm,
    top=2cm,
    bottom=2cm,
    marginparwidth=1.5cm,
    headheight=2cm
}

\usepackage{array}
\usepackage{makecell}





%Task and support template

%\begin{table}[htbp]
%    \centering
%    \begin{tabular}{|p{5cm}|p{10cm}|}
%        \hline
%        \multicolumn{2}{|c|}{\makecell[cl]{\textbf{Task:} \\
%        \textbf{Purpose:} \\
%        \textbf{Frequency:}}}\\
%        \hline
%        % Row 1
%        Sub task 1 & Example Solution 1 \\
%        \hline
%        % Row 2
%        ? & ? \\
%        \hline
%        % Row 3
%        ? & ? \\
%        \hline
%        % Row 4
%        ? & ? \\
%        \hline
%        \multicolumn{2}{|c|}{\makecell[cl]{\textbf{Variant:}}}\\
%        \hline
%        ? & ? \\
%        \hline
%    \end{tabular}
%    \caption{Example Table}
%    \label{tab:example}
%\end{table}



\title{Software Requirements Specification
Document\\
Cosy Koala IT}

\author{
  Vincent, Malzinskas\\
  \texttt{6474322}
  \and
  Aashim Lal, Memanaparambil Asokalal\\
  \texttt{103794571}
  \and
  Jordan, Zaz\\
  \texttt{6386601}
  \and
  Julian, Lai\\
  \texttt{103594920}
}

\date{March 2024}

\begin{document}


\maketitle
\newpage
\tableofcontents
\newpage


\section{Introduction}
This document contains the specifications for a operational information system (OIS) for the Cosy Koala restaurant in Hawthorn, Vic. This document will outline the specifications, requirement and quality attributes of the operational information system

\subsection{Purpose}
The purpose of this document is to detail the functionalities and features of the OIS, aiming to streamline various restaurant operations as the business grows. From customer management and interactions to internal services and task management, the goal of this system is to allow the Cosy Koala to operate at a targeted concurrent customer patronage of 150 guests.

\subsection{Scope}
The OIS will facilitate the following high level tasks:
\begin{enumerate}
    \item Customer Management.
    \item Customer Interaction.
    \item Internal Services.
\end{enumerate}


\section{Project Overview}
The Cosy Koala currently has a maximum capacity of 50 customers at any one time. It is making a physical expansion on premises to sit 150 customers. The current technical integration with operations is low. Orders from guests, interactions with the kitchen, and accounting are all done manually.
The Cosy Koala has identified multiple ideas they would like to include in the OIS:
\begin{enumerate}
    \item The new system shall support reservations.
    \item The new system shall support taking orders from customers.
    \item The new system shall support information sharing with the Kitchen.
    \item The new system shall support creating invoices.
    \item The new system shall support creating receipts for customers.
    \item The new system shall support handling payments.
    \item The new system shall support basic statistics about ordered menu items.
    \item The new system shall support online availability of the menu.
    \item The new system shall support ordering from online take-away menus.
    \item The new system shall possibly support arranging delivery.
\end{enumerate}

\subsection{Domain Vocabulary}
\begin{enumerate}
    \item OIS : Operation Information System.
    \item FOH : Front of house.
    \item BOH : Back of house.
\end{enumerate}
\subsection{Pain Points}
\begin{enumerate}
    \item Customer information:
    \begin{enumerate}
        \item Managing 50+ customer orders.
        \item Taking onsite orders.
        \item Taking offsite orders.
        \item Collecting data on sales.
        \end{enumerate}
    \item Front to back of house interactions:
    \begin{enumerate}
        \item Organize orders.
        \item Make sure food is sent to correct customer.
    \end{enumerate}
    \item Administrative functions:
    \begin{enumerate}
        \item Gather customer data.
        \item Update website with marketing data.
        \item Analyse customer data.
        \item Provide offsite order and payment.
        \item Provide delivery.
    \end{enumerate}
\end{enumerate}


\subsection{Domain Entities}
\begin{enumerate}
    \item Meal
    \item Payments
    \item Customer
    \item Front of House Staff
    \item Seat
    \item Table
    \item Dining Room
    \item Receipts
    \item Sales Data
    \item Order
    \item Marketing Data, i.e. Menu lists, JPGs etc.
    \item Point of Sales
    \item Kitchen
\end{enumerate}


\subsection{Actors}
\subsection{Identifying Actors}

As Software engineers we have a deep understanding of what is needed to create a successfully, effective and functional application; Versus what the client thinks the software needs in order to have those qualities. This statement implies we have the knowledge and skills to design and execute the development of some software, but a client or a group of stakeholders may have some tacit understanding of what needs to be done, which may have some merit, but is ultimately uniformed and may be inaccurate. By that same token, we software engineers may also have a tacit understanding of the roles of the client and stake holders. We attempt to bridge the gap by presenting some ideas and questions of what the client may need, what we expect of them and what their roles entail. 

The first approach is coming up with a list of Actors; Defined as broad group of entities interacting within our business domain, like stakeholders, authorities and land-lords. We also list some functionality we assume they need from the program.

In this Scenario the team has identified , the stake holder in Koala Cafe to be:


\begin{enumerate}
    \item Onsite Customer
    \item Offsite Customer
    \item Front of House Staff Member
    \item Back of House Staff Member
    \item Admin
    \item Website
    \item Delivery System
    \item EFTPOS or Banking System
\end{enumerate}

\subsection{Task - Brief}
\begin{enumerate}
    \item Take onsite customer order .
    \item Take offsite customer order.
    \item Take onsite customer payment.
    \item Take offsite customer payment.
    \item Create offsite customer receipt.
    \item Create onsite customer receipt.
    \item Communicate order with Kitchen.
    \item Seat customer.
    \item Drop food to customer.
    \item Deliver food to customer.
\end{enumerate}

\subsection{Stakeholders}
\begin{itemize}
    \item {Business Operations}
    \begin{itemize}
        \item {Finances}
        \item {Roster}
        \item {Web Portal}
        \item {Admin Privileges}
        \item {Marketing}
        \item {Statistics/Analytics}
        \item Pay bills
        \item Pay Salary
    \end{itemize}
    \item {Management}
    \begin{itemize}
        \item {Set Roster}
        \item {Alter Menu}
        \item {Alter table data}
        \begin{itemize}
            \item {Arrange seating positions}
            \item {Alter table numbers}
        \end{itemize}
        \item {Approve Refunds}
        \item {Order Stock}
        \item {Have access to wait staff functionality}
        \item View feedback
    \end{itemize}
    \item {Barista}
    \begin{itemize}
        \item {Request Stock}
        \item view and print coffee orders
        \item alert stock running low
        \item {Have access to wait staff functionality}
    \end{itemize}
    \item {Wait Staff}
    \begin{itemize}
        \item {Take Order}
        \item {Process Payment}
        \item {Reserve Table}
        \item {Special meal Requests}
        \item See roster
        \item Record feedback
        \item Receive Tips
    \end{itemize}
    \item {Chef}
    \begin{itemize}
        \item Alter Menu
        \item Alter Menu Items
        \item Receive Order and print ticket
        \item Confirm Order
        \item Order Up
        \item Order Stock
        \item Alter kitchen roster
        \item View Feedback
    \end{itemize}
    \item {Kitchen Staff}
    \begin{itemize}
        \item Request stock
        \item Request Supply's
        \item See Roster
        \item Alert stock running low
    \end{itemize}
    \item {Maintenance}
    \begin{itemize}
        \item Order Supplies
        \item track maintenance tasks
        \item alter maintenance to-do list
    \end{itemize}
    \item {Customer}
    \begin{itemize}
        \item make booking online or via phone
        \item order online via phone
        \item order or book via App
        \item order via QR code
        \item order via Digital Kiosk
        \item 
    \end{itemize}
\end{itemize}
Other Actors:
\begin{itemize}
    \item {Health and Safety}
    \item {Land Lord}
    \item {Suppliers}
\end{itemize}


\subsection{Project Goals \normalsize\textbf{Author: Vincent}}
\subsubsection{Primary Goals}
The primary Goal of Cosy Koala and the reason for releasing a tender for an OIS software provider is to be able to support a physical increase of max guest numbers from 50 to 150 in their restaurant.
\subsubsection{Secondary Goals}
\begin{enumerate}
    \item Support increased business through take-away.
    \item Support informed customers through their website.
    \item Support informed management about the customers purchases.
    \item Capture potential customers through their website.
\end{enumerate}
\subsubsection{Tertiary Goals}
Cosy Koala is interested in the possibility of supporting the arrangement of delivery.

\subsection{Assumptions \normalsize\textbf{Author: Vincent}}
A number of assumptions are made in relation to the implementation of this OIS is that manual scaling of operations are impossible. Whether that means no more staff will be available. Or whether staff will be expected to have other duties.

The system should model closely as possible to the current practices.


\clearpage

\section{Problem Domain}
It is helpful to think of Data at the Cosy Koala in three ways. Reading data in the restaurant system. Writing data to the restaurant system. Or writing data to the website system. 
Many of the tasks the Cosy Koala needs to complete fit into one of these three Domain Data Models in a general way. 
Example: Taking a customers order fits into a section of the "Writing data to the restaurant system" domain model.


The reading data from the website is not included as this is typically accessed through a website in a standard way.

\clearpage
\subsection{ Domain Data model Reading \normalsize\textbf{Author: Vincent}}
Reading Data from the Cosy Koala Restaurant Systems. The green rectangles are actors.
The front of house actor accesses restaurant system data through the " Point of sales machine".
Admin accesses restaurant system data through their GUI likely from a personal computer.
Back of house only really needs to read data relevant to orders at almost all times. So they access the the restaurant system exclusively through the Docket machine and order dockets.





\begin{figure}[!ht]
    \centering
    \includegraphics[width=15cm]{Domain Data Model access.jpg}
    \caption{Cosy Koala Domain Model Reading.}
    \label{fig:Domain_Model_access}
\end{figure}

\clearpage
\subsection{Cosy Koala Domain Model Writing.}
Here we model how many task involving moving data into the restaurant system work.

Example: The task of taking orders from a customer involve the guest making orders and building a meal which is passed to the waiter (Front of house member) who then puts it into the restaurant system via the point of sales machine.
\begin{figure}[!ht]
    \centering
    \includegraphics[width=15cm]{Domain Data Model input.jpg}
    \caption{Cosy Koala Domain Model Writing.}
    \label{fig:Domain_Model_input}
\end{figure}

\clearpage
\subsection{Cosy Koala Domain data Model Website Writing}
The third major domain data model regards the writing of information from the restaurant system to the website.

Example: The admin staff member will access the data from the restaurant system before analyzing it and generating marking data which they then model the information they want to put to the website.
\begin{figure}[!ht]
    \centering
    \includegraphics[width=15cm]{Domain Model Data input website.jpg}
    \caption{Cosy Koala Domain Model Website Writing.}
    \label{fig:Domain_Model_website}
\end{figure}


\subsubsection{Entity Descriptions - Detailed}
\textbf{Meal:} To group together items for payment, delivery, or dropping at a table it would be helpful to place them into a unit of a meal. This way a meal can be paid for as one, or moved to a new table as one etc.

\textbf{Payments:} This will model the owed amounts for a meal, capable of of dividing payments based on a number of divisions, i.e. by item, by table, by room for functions. It will interact with receipts and and the Payment System (Actor).

\textbf{Customer:} This will model both the onsite and offsite customers, it will interact with bookings, tables, dining rooms, sales Data. It will contain customer specific data like meal, sales data.

\textbf{Front of House:} This will be a model for anyone interacting with customers capable of creating orders, bookings, taking payment and issuing receipts etc.

\textbf{Seat:} A customer can be assigned a seat to locate a meal or part of a meal to the correct customer.

\textbf{Table:} Contains seats and can be used for grouping meals into a larger single unit for booking or payment.

\textbf{Dining Room:} Containerizes and organizes customers orders i.e. If the restaurant has a multiple rooms this can be used to better locate customers for orders. It will contain tables which contain seats and customers.

\textbf{Receipts:} Will be a record for Sales data and Customers of a meal including its expense, items, time, location. Will be issues by the payment entity.

\textbf{Sales Data:} This is a aggregation of the Receipts data.

\textbf{Order:} (Onsite) Contains the meal order items, seat, table, dining room, that is ordered by a customer. \\(Offsite) Contains the meal order items, customer that is ordered by a customer.

\textbf{Marketing Data:} This will be the data that can be sent to the Website (Actor) in order to update the website.

\textbf{Point of Sales Machine:} Facilitates payment transactions. Issues receipts. Logs sales data. Is operated be a front of house staff member. Interacts with the payment entity and the payment system (Actor)

\textbf{Kitchen:} Containerizes and organizes customers food order for pick up by front of house.



\subsection{Actors Descriptions - Detailed}

\textbf{Onsite Customer:} Is a person dining on food or beverages prepared by Cosy Koala. This is done on site in a dining room at a table. .

\textbf{Offsite Customer:} Is a person dining on food or beverages prepared by Cosy Koala. This is done off site at any location other than Cosy Koala dining rooms.

\textbf{Front of House:} This is a staff member who has direct contact with the customer. They take orders from customers. Take orders to the Kitchen. Bring customers their food. They are responsible for maintaining the orders integrity (Making sure the customer gets what they pay for). The take customer transactions. Communicate with Admin.

\textbf{Back of House:} Prepare the customers food order. Responsible for making sure front of house receive the food order with correct information i.e. order number etc. Communicate with Admin.

\textbf{Website:} Displays the restaurant information for customers.

\textbf{Admin:} Get the logged data from the Point of Sales Machine.
Generate analytics from the Sales data.
Web Site: Display general information about the restaurant.

\textbf{EFTPOS or Banking System:} This is the outside system that will have to interact with the POS entity in order to facilitate payments.

\textbf{Delivery System:} This the outside system that facilitates delivery, like Uber, or menulog etc.


\clearpage
\subsection{Workflows \normalsize\textbf{Author: Vincent}}

\begin{table}[htbp]
    \centering
    \begin{tabular}{|p{5cm}|p{10cm}|}
        \hline
        \multicolumn{2}{|c|}{\makecell[cl]{\textbf{Task: Administrate the restaurant} \\
        \textbf{Actor: Admin} \\
        \textbf{Purpose:} To manage the technical administration tasks so that everything from bookings\\
        to invoices are completed in a timely correct manner.}}\\
        \hline
        % Row 1
        Sub task  & Example Solution  \\
        \hline
        % Row 2
        1. Review past data, including sales, costs, etc, particular attention is paid to the past day of service.
        
        \textbf{Problem:} Records are not accurate. & Communicate directly with staff from the previous day about the data generated. Automate data entry. \\
        \hline
        % Row 3
        2. Update restaurant bookings, and communicate with customers. 
        
        \textbf{Problem:} Bookings are unavailable for guest. & Communicate with the guest about an alternative date to book. Make the available dates and services on a given date accessible to customers. \\
        \hline
        % Row 4
        3. Communicate and inform the FOH and BOH staff about upcoming customer demands, bookings and general workload. 
        
        \textbf{Problem:} The bookings are changing rapidly or give short notice & Auto update FOH and BOH staff about bookings etc. \\
        \hline
    \end{tabular}
    \caption{Workflow: 1}
    \label{tab:example_wf_1}
\end{table}

\clearpage
\begin{table}[htbp]
    \centering
    \begin{tabular}{|p{5cm}|p{10cm}|}
        \hline
        \multicolumn{2}{|c|}{\makecell[cl]{\textbf{Task: Prepare the kitchen} \\
        \textbf{Actor: Back of House} \\
        \textbf{Purpose:} To prepare enough food or drink for the upcoming guest, planning at the service, \\day and week level is appropriate.}}\\
        \hline
        % Row 1
        Sub task  & Example Solution  \\
        \hline
        % Row 2
        1. Review bookings for today and tomorrow in close detail as well as a week ahead in brief detail. 
        
        \textbf{Problem:} Records are not accurate. Bookings have change but kitchen doesn't know. & Automate directly live data to the kitchen. \\
        \hline
        % Row 3
        2. Receive orders and cook food. 
        
        \textbf{Problem:} Orders are not accurate to items and customers & Deliver orders to kitchen directly from order entry at POS machine, - no hand written dockets. \\
        \hline
        % Row 4
        3. Order needed food for upcoming bookings. 
        
        \textbf{Problem:} Bookings have not been updated. & Automate directly live data to the kitchen. \\
        \hline
    \end{tabular}
    \caption{Workflow: 2}
    \label{tab:example_wf_2}
\end{table}


\clearpage
\begin{table}[htbp]
    \centering
    \begin{tabular}{|p{5cm}|p{10cm}|}
        \hline
        \multicolumn{2}{|c|}{\makecell[cl]{\textbf{Task: Receive and dine guests} \\
        \textbf{Actor: Front of House} \\
        \textbf{Purpose:} To prepare the rooms for upcoming guest, planning at the service\\ and day is appropriate.}}\\
        \hline
        % Row 1
        Sub task  & Example Solution  \\
        \hline
        % Row 2
        1. Review bookings for today in close detail. 
        
        \textbf{Problem:} Records are not accurate. Bookings have change but front of house staff don't know. & Automate directly live data to the front of house staff. \\
        \hline
        % Row 3
        2. Set tables and places for bookings and expected walk ins. 
        
        \textbf{Problem:} Bookings have change but front of house staff don't know. & Automate directly live data to the front of house staff. \\
        \hline
        % Row 4
        3. Receive guest/s. Take to booked seating.
        
        \textbf{Problem:} Guest/s has not booked & Find the available tables and seats.\\
        \hline
        4. Take orders from guests. 
        
        \textbf{Problem:} The guest changes meal order, guest changes table or seat, guests leave, more guest come. & Have flexible table and meal placement that can be altered.\\
        \hline
        5. Take payment. 
        
        \textbf{Problem: } Payment is to be split by meal, item, guest, table or some other way. & Make a flexible payment option.\\
        \hline
    \end{tabular}
    \caption{Workflow: 3}
    \label{tab:example_wf_3}
\end{table}



\clearpage
\subsection{Tasks}

\subsubsection{Work Area 1: Front of House}
Areas: Dinning Hall, Coffee station, Point of sales
Greet guests, seat them, bring water to table, prepare drinks. 
Standing, assume medium-high frequency of traffic.
Have responsibility over several tables occasionally helping when larger party's enter,
Users: IT Novice, high school level education

\subsubsection{Work area 2: Back of House}
Area: Kitchen, Dry stock, Cool room, cleaning area
Prepare Orders, Meal prep, Receive Deliveries, Specials
Standing, fast past - dynamic environment, working in a team of 3-6 people, long shifts(8-12 hours)
Have a variety of rolls in the kitchen, from cleaning to cooking
Users: IT Novice, high school level education

\subsubsection{Work area 3: Business, Finance and Administration}
Area: Office
Revenue, Spending, Ordering, Approving Changes, Web, Marketing, Hiring, Roster, Stats
Small team, ad-hoc support from external services.
Usually outsourcing work or working alone or in pair post breakfast/lunch rush
Users: IT intermediate:, college level education


\subsection{Complete Payment transaction onsite.}
\begin{table}[htbp]
    \centering
    \begin{tabular}{|p{5cm}|p{10cm}|}
        \hline
        \multicolumn{2}{|c|}{\makecell[cl]{\textbf{Task:} Complete Payment transaction onsite. \\
        \textbf{Purpose:} Take the money owed to the restaurant after a sale of a food or drink and issue receipt.\\
        \textbf{Frequency:} Up to 150 times a breakfast, lunch or dinner service.}}\\
        \hline
        % Row 1
        \textbf{Sub task} & \textbf{Example Solution} \\
        \hline
        % Row 2
        Locate the meal in system. & Enter the table number, and dining room name in order to find the meal. \\
        \hline
        % Row 3
        Inform customer of details of meal, price etc. & The GUI can display relevant meal information. \\
        \hline
        % Row 4
        Initiate payment system. & FoH Staff member can send the price of the meal to the EFTPOS machine. \\
        \hline
        % Row 5
        Display whether the payment was successful. & The GUI can display to the customer and the staff member whether payment was successful. \\
        \hline
        % Row 6
        Offer to print receipt. & The GUI could display an option for the FoH or Customer to select to print a receipt. \\
        \hline
        Print receipt & System send information to printer, possibly by network. \\
        \hline
        Move transaction to new state - payed & The system can validate the transaction was successful and move the meal to a payed/cleared data storage. \\
        \hline
        \multicolumn{2}{|c|}{\makecell[cl]{\textbf{Variant:}}}\\
        \hline
        Customer only wants to pay for their meal and a table with multiple people & System should allow the FoH to select items from a meal for individual payment. \\
        \hline
    \end{tabular}
    \caption{Taking Payment}
    \label{tab:Complete Payment transaction onsite.}
\end{table}

\clearpage
\subsection{Take a booking}
\begin{table}[htbp]
    \centering
    \begin{tabular}{|p{5cm}|p{10cm}|}
        \hline
        \multicolumn{2}{|c|}{\makecell[cl]{\textbf{Task:} Take a customers booking. \\
        \textbf{Purpose:} To pre-allocate a customer to a table in order to make sure the customer can dine when arriving. \\
        \textbf{Frequency:} Up to 150 time a service, 3 services a day.}}\\
        \hline
        % Row 1
        \textbf{Sub task} & \textbf{Example Solution} \\
        \hline
        % Row 2
        Receive customers correspondance - email, phone call & Each day the admin will check the restaurant email and collect any bookings that have been sent.  \\
        \hline
        % Row 3
        For each booking review availability. & If some one wants to book a table for 4 at 8pm on Friday, the admin will first check if there is space for a table of 4 at that date and time. \\
        \hline
        % Row 4
        Enter booking into record. & The admin will make a recording of the booking so that the resources cannot be double booked. \\
        \hline
        Inform the Front of house staff & The admin will transfer the bookings for today to a sheet for the front of house to organise the dining room to host. \\
        \hline
        Return confirmation correspondence to the customer & The admin will send a return email with the confirmation details.\\
        \hline
        \multicolumn{2}{|c|}{\makecell[cl]{\textbf{Variant:}}}\\
        \hline
        The Admin receives a phone call. & This booking will need to be reviewed on the spot in order to make confirmation. \\
        \hline
        The booking is made on short notice & After the bookings and confirmation have already been made for the day. A new booking comes in. The admin will need to directly confirm and inform front of house to accommodate the booking. \\
        \hline
        The booking cannot be made. & The admin must reject the booking and offer a possible alternative.\\
        \hline
    \end{tabular}
    \caption{Take a booking}
    \label{tab:Take a booking}
\end{table}

\clearpage
\subsection{Take a tables order}
\begin{table}[htbp]
    \centering
    \begin{tabular}{|p{5cm}|p{10cm}|}
        \hline
        \multicolumn{2}{|c|}{\makecell[cl]{\textbf{Task:} Take a tables order. \\
        \textbf{Purpose:} To get orders from a guest or group of guests at a table, enter them into the system, so that the order can be shared with the kitchen and progress can be tracked. Including verifying that the order has been completed.  \\
        \textbf{Frequency:} If the restaurant has only single guest tables this could occur 150 times a service, However, a more likely number is around 70 times a service 3 services a day}}\\
        \hline
        % Row 1
        Sub task  & Example Solution  \\
        \hline
        % Row 2
        Receive customer requests & Locate menu item in app/website using the search function. \\
        \hline
        % Row 3
        Confirm item variant (sauces, size, etc) or inform sell out & Check menu item in app/website, it will be marked with a red X if it is sold out, and have options in a drop down menu if the particular menu item has variants. \\
        \hline
        % Row 4
        Send order to kitchen  & Press the send button to send an alert with all the details to the kitchen. \\
        \hline
        \multicolumn{2}{|c|}{\makecell[cl]{\textbf{Variant:}}}\\
        \hline
        ? & ? \\
        \hline
    \end{tabular}
    \caption{Take a tables order}
    \label{tab:Take a tables order}
\end{table}

\clearpage
\subsection{Cook Food}
\begin{table}[htbp]
    \centering
    \begin{tabular}{|p{5cm}|p{10cm}|}
        \hline
        \multicolumn{2}{|c|}{\makecell[cl]{\textbf{Task:} Cook Food for guests. \\
        \textbf{Purpose:} This task is for the kitchen and the back of house staff to receive a tables order with all the information needed in order to properly cook it as requested in an organized manor, i.e. Allergy information, time order was taken, guests at table, menu item, course (entree).  \\
        \textbf{Frequency:} If the restaurant has only single guest tables this could occur 150 times a service, However, a more likely number is around 70 times a service 3 services a day}}\\
        \hline
        % Row 1
        Sub task  & Example Solution  \\
        \hline
        % Row 2
        Receive notification from wait staff & A screen (computer) displays the new order alongside existing ones. \\
        \hline
        % Row 3
        Start cooking the order  & Mark the order as in progress on the screen. \\
        \hline
        % Row 4
        Finish cooking and plate up & Mark the order as done on the screen, which sends an alert to FOH. \\
        \hline
        \multicolumn{2}{|c|}{\makecell[cl]{\textbf{Variant:}}}\\
        \hline
        ? & ? \\
        \hline
    \end{tabular}
    \caption{Cook Food}
    \label{tab:Cook Food}
\end{table}

\clearpage
\subsection{Create Analytics}
\begin{table}[htbp]
    \centering
    \begin{tabular}{|p{5cm}|p{10cm}|}
        \hline
        \multicolumn{2}{|c|}{\makecell[cl]{\textbf{Task:} Create analytics from restaurant data. \\
        \textbf{Purpose:} The admin staff member gathers the sales information from the restaurant system in order to discover and statistical information about what sells well, and what the restaurant can do to improve business. This uses analytical technique. \\
        \textbf{Frequency:} At a minor level daily and weekly, at significant level probably quarterly.}}\\
        \hline
        % Row 1
        Sub task  & Example Solution  \\
        \hline
        % Row 2
        Record all details  & Enter in data regularly  \\
        \hline
        % Row 3
        Generate analytics  & Press generate report button to generate graphs and reports. \\
        \hline
        % Row 4
        ? & ? \\
        \hline
        \multicolumn{2}{|c|}{\makecell[cl]{\textbf{Variant:}}}\\
        \hline
        ? & ? \\
        \hline
    \end{tabular}
    \caption{Create Analytics}
    \label{tab:Create Analytics}
\end{table}

\clearpage
\subsection{Update Website}
\begin{table}[htbp]
    \centering
    \begin{tabular}{|p{5cm}|p{10cm}|}
        \hline
        \multicolumn{2}{|c|}{\makecell[cl]{\textbf{Task:} Update website restaurant details. \\
        \textbf{Purpose:} Update the Website with details about menu and restaurant menu offerings, including advertising things like seasonal specials, or discounts, and even opening hours for special events  \\
        \textbf{Frequency:} At most weekly, more likely every few weeks.}}\\
        \hline
        % Row 1
        Sub task  & Example Solution  \\
        \hline
        % Row 2
        Access admin section of the website & Input admin credentials  \\
        \hline
        % Row 3
        Make relevant changes  & Edit menu items and change advertisements using drag and drop feature, change prices by typing in a new price \\
        \hline
        % Row 4
        Save and update the website  & Push the changes to the live website. \\
        \hline
        \multicolumn{2}{|c|}{\makecell[cl]{\textbf{Variant:}}}\\
        \hline
        ? & ? \\
        \hline
    \end{tabular}
    \caption{Update Website}
    \label{tab:Update Website}
\end{table}

\clearpage
\subsection{Drops food to table}
\begin{table}[htbp]
    \centering
    \begin{tabular}{|p{5cm}|p{10cm}|}
        \hline
        \multicolumn{2}{|c|}{\makecell[cl]{\textbf{Task:} To take food from the kitchen when ready and deliver it to guest at the table \\
        \textbf{Purpose:} The kitchen will cook the food it receives orders for with the intention to complete the cooking of a tables order together i.e. entrees with entrees. A front of house member will then take the food to the correct customer in the restaurant. \\
        \textbf{Frequency:} Up to 150 times a service, at 3 services a day.}}\\
        \hline
        % Row 1
        Sub task & Example Solution  \\
        \hline
        % Row 2
        Collect food from kitchen  & Set status of order as delivered using phone/ipad. \\
        \hline
        % Row 3
        Take order to customers  & Use phone/ipad to check order information for table number. \\
        \hline
        % Row 4
        ? & ? \\
        \hline
        \multicolumn{2}{|c|}{\makecell[cl]{\textbf{Variant:}}}\\
        \hline
        ? & ? \\
        \hline
    \end{tabular}
    \caption{Drops food to table}
    \label{tab:Drops food to table}
\end{table}

\clearpage
\subsection{Share information Admin-Front of house, Admin-Back of house.}
\begin{table}[htbp]
    \centering
    \begin{tabular}{|p{5cm}|p{10cm}|}
        \hline
        \multicolumn{2}{|c|}{\makecell[cl]{\textbf{Task:} Share information Admin to front of house, and admin to back of house. \\
        \textbf{Purpose:} The admin need to share details of bookings and customer or guests with both the front of house and back of house. These two groups need to share information of capabilities, staff numbers, and the state of the restaurant and kitchen.  \\
        \textbf{Frequency:} At least once a day, sometimes maybe as many as 5 times a day.}}\\
        \hline
        % Row 1
        Sub task  & Example Solution  \\
        \hline
        % Row 2
        Send out information  & Type information into computer internal mail then send it to all/relevant employees who will receive an alert on their tablet or phone \\
        \hline
        % Row 3
        ? & ? \\
        \hline
        % Row 4
        ? & ? \\
        \hline
        \multicolumn{2}{|c|}{\makecell[cl]{\textbf{Variant:}}}\\
        \hline
        ? & ? \\
        \hline
    \end{tabular}
    \caption{Admin communication}
    \label{tab:Admin communication}
\end{table}

\clearpage
\subsection{}
\begin{table}[htbp]
    \centering
    \begin{tabular}{|p{5cm}|p{10cm}|}
        \hline
        \multicolumn{2}{|c|}{\makecell[cl]{\textbf{Task:} \\
        \textbf{Purpose:} \\
        \textbf{Frequency:}}}\\
        \hline
        % Row 1
        Sub task 1 & Example Solution 1 \\
        \hline
        % Row 2
        ? & ? \\
        \hline
        % Row 3
        ? & ? \\
        \hline
        % Row 4
        ? & ? \\
        \hline
        \multicolumn{2}{|c|}{\makecell[cl]{\textbf{Variant:}}}\\
        \hline
        ? & ? \\
        \hline
    \end{tabular}
    \caption{Example Table}
    \label{tab:example}
\end{table}






\subsection{Functional Requirements and Task Descriptions}
\begin{figure}[!ht]
    \centering
    \includegraphics[width=15cm]{Domain_product_diagram.jpg}
    \caption{Cosy Koala Domain Requirements}
    \label{fig:Domain_Product}
\end{figure}

\clearpage
\subsection{CRUD}

\begin{table}[h!]
  \centering
  \begin{tabular}{|l|c|c|c|c|c|c|}
    \hline
    \textbf{Tasks / Entities} & \textbf{Customer} & \textbf{Table} & \textbf{Stock} & \textbf{Kitchen} & \textbf{Invoice and receipts} & \textbf{Statistics and reports} \\
    \hline
    Create reservation & CU & U & & & & \\
    \hline
    Take order & U & U & R & U & U & \\
    \hline
    Inform kitchen of order & & & & U & & \\
    \hline
    Create invoice and receipts & U & U & & & R & \\
    \hline
    Handle payments & & & & & U & \\
    \hline
    Analysing statistics on ordered menu items & RU & & R & R & R & CU \\
    \hline
    Complete payment transaction onsite & U & & & & CRU & U \\
    \hline
    Cook food & & RU & RUD & CRUD & & \\
    \hline
    Create Analytics & R & R & R & R & R & CRUD \\
    \hline
    Update Website & & & R & & & R \\
    \hline
    Drop food to table & RU & RUD & & RUD & U & \\
    \hline
    Share information Admin-Front of house, admin-back of house & R & R & R & R & R & R \\
    \hline
  \end{tabular}
  \caption{Your table title here}
\end{table}

\clearpage
\section{Quality Attributes of System}

\subsection{Usability} 
Goal: Ensure that the system is intui ve and easy to use for all users, including staff, 
customers, and management. 
Detailed Descrip on: The interface should be clear and straigh orward, allowing users with 
minimal technical skills, such as children or elderly individuals, to navigate and use the 
system with ease. For example, a child with basic phone usage experience should be able to 
place an order through a digital menu without difficulty. This can be achieved through, 
simple menus, clear labels, and helpful icons. The system should also provide accessible help 
resources and tool ps. 
Real-Life Reference:  
\subsection{Security}
Goal: Protect sensi ve data and ensure that transac ons are secure. 
Detailed Descrip on: Implement industry-standard security measures such as SSL/TLS for 
data transmission, encrypted storage for sensi ve informa on, and regular security audits. 
The system should have role-based access controls to ensure that only authorized personnel 
can access certain func onali es. For example, only managers should be able to access 
financial reports or employee records and on the other hand it should also make sure that 
the customers payment details are encrypted and not accessible even by the managers. 
Real-Life Reference: 
\subsection{Correctness}
Goal: Ensure the system processes informa on accurately and performs its func ons 
correctly. 
Detailed Descrip on: The system should validate input data to prevent errors and ensure 
the accuracy of orders, billing, and inventory management. Automated tes ng and quality 
assurance processes should be in place to detect and correct errors before deployment. For 
example, when a customer places an order, the system should accurately reflect the chosen 
items, quan es, and prices in the final bill. 
Real-Life Reference:  
\subsection{Reliability (Availability)} 
Goal: Ensure the system is opera onal and available during business hours without 
unexpected down mes. 
Detailed Descrip on: The system should be designed for high availability with failover 
mechanisms and redundancy to handle peak mes and unexpected loads. For example, 
during a busy dinner service, the system should remain func onal and responsive, managing 
mul ple orders and reserva ons simultaneously without crashing or slowing down. 
Real-Life Reference:  
\subsection{Performance}
Goal: Ensure the system responds quickly to user interac ons and processes data efficiently. 
Detailed Descrip on: The system should be opmized for fast response mes, even under 
heavy load. This includes efficient database queries, opmized code, and the ability to 
handle concurrent users smoothly. For instance, when mul ple customers place orders 
simultaneously, the system should process these quickly without delays or bo lenecks. 
Real-Life Reference:  





\section{Other Requirements}
\subsection{Acquiring Knowledge using a Tacit Questionnaire}(Quality Requirements)
Careful construction of a tacit questionnaire allows us to extract data from stakeholders without requiring them to understand deep technical details. Instead, we are more focused on discovering what they expect or would like from the system, drawing from their experience and implicit knowledge of their role within their domain.

Here we are looking to extract some data about the and functional requirements





\section{Validation of Requirements}
\subsection{Development Approach: Working with the Client}
Ensuring that we closely align with a client's requirements at the early stages of the software development life-cycle is crucial in creating a robust and thoughtful software architecture. We are mindful that any problems faced now are much easier to tackle earlier rather than later on when they are "baked-in." Any changes that are made to later in the software's development can be costly and time consuming since some systems are likely to need re-programming or re-designing which can have a cascading effect on other system, which can lead to a larger restructuring. 

The back and forth between the two parties; the client and producer (us) is a proven methodology (Validation) for creating a software system's architecture as we invest more time early to iron out any projected issues in partnership with the client. We consider all levels of domains: societal, environmental, global,business,customer and methodically come up with strategies that we come up tacitly from a high level and systematically decompose the ideas from the abstract all the way down to the software's design. Source: IEEE's standards for software architecture documentation and the Software Engineering Institute's (SEI) Architecture Trade-off Analysis Method (ATAM).

The key is EARLY Validation.


\section{Possible Solutions}


\end{document}